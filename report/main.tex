%%%%%%%%%%%%%%%%%%%%%%%%%%%%%%%%%%%%%%%%%% University Assignment Title Page 
% LaTeX Template
% Version 1.0 (27/12/12)
%
% This template has been downloaded from:
% http://www.LaTeXTemplates.com
%
% Original author:
% WikiBooks (http://en.wikibooks.org/wiki/LaTeX/Title_Creation)
%
% License:
% CC BY-NC-SA 3.0 (http://creativecommons.org/licenses/by-nc-sa/3.0/)
% 
% Instructions for using this template:
% This title page is capable of being compiled as is. This is not useful for 
% including it in another document. To do this, you have two options: 
%
% 1) Copy/paste everything between \begin{document} and \end{document} 
% starting at \begin{titlepage} and paste this into another LaTeX file where you 
% want your title page.
% OR
% 2) Remove everything outside the \begin{titlepage} and \end{titlepage} and 
% move this file to the same directory as the LaTeX file you wish to add it to. 
% Then add \input{./title_page_1.tex} to your LaTeX file where you want your
% title page.
%
%%%%%%%%%%%%%%%%%%%%%%%%%%%%%%%%%%%%%%%%%
%\title{Title page with logo}
%----------------------------------------------------------------------------------------
%	PACKAGES AND OTHER DOCUMENT CONFIGURATIONS
%----------------------------------------------------------------------------------------

\documentclass[12pt, french]{article}
\usepackage[utf8]{inputenc} % remove x to make it compatible to biblography
\usepackage[T1]{fontenc} % added instead of utf8x
\usepackage{textcomp} % added instead of utf8x
\usepackage{babel}

\usepackage{amsmath}
\usepackage{graphicx}
%\usepackage{hyperref} % commented and use fix below instead
\usepackage[colorinlistoftodos]{todonotes}
\usepackage{caption}
\usepackage{verbatim}
\usepackage{ragged2e}
\usepackage[hyphens]{url} % quick and dirty fix to the overflowing url problem
%\usepackage{biblatex}
\usepackage{csquotes}
\usepackage[backend=biber,style=alphabetic,sorting=nyt]{biblatex}
\addbibresource{sources.bib}
\justifying
\renewcommand*\contentsname{Table des matières}
\begin{document}

\begin{titlepage}

\newcommand{\HRule}{\rule{\linewidth}{0.5mm}} % Defines a new command for the horizontal lines, change thickness here

\center % Center everything on the page
 
%----------------------------------------------------------------------------------------
%	HEADING SECTIONS
%----------------------------------------------------------------------------------------

\textsc{\LARGE Data Science Semester Project}\\[1.5cm] % Name of your university/college
\textsc{\large{Author: }Julien Heitmann}\\[0.5cm] % Major heading such as course name

%----------------------------------------------------------------------------------------
%	TITLE SECTION
%----------------------------------------------------------------------------------------

\HRule \\[0.6cm]
{ \huge \bfseries Neural network pruning and weight geometry}\\[0.5cm] % Title of your document
\HRule \\[1.5cm]
 
%----------------------------------------------------------------------------------------
%	AUTHOR SECTION
%----------------------------------------------------------------------------------------


% If you don't want a supervisor, uncomment the two lines below and remove the section above
%\Large \emph{Author:}\\
%John \textsc{Smith}\\[3cm] % Your name

%----------------------------------------------------------------------------------------
%	DATE SECTION
%----------------------------------------------------------------------------------------
\begin{figure}[!h] 
        \centering \includegraphics[width=1\columnwidth]{images/EPFL_Logo_Digital_RGB_PROD.jpg}
\end{figure}

\vfill % Fill the rest of the page with whitespace
\end{titlepage}
\tableofcontents
\newpage
\justify

\section{Introduction}

\section{Contexte historique}

\subsection{De la révolution culturelle au lancement du premier satellite}

\subsection{Réformes et modernisation du programme spatial chinois}

\subsection{La Chine s'affirme comme puissance spatiale majeure}

\section{Objectifs et enjeux du programme spatial}

\subsection{Économie}

\subsection{La conquête spatiale, au service de l'image de la Chine}

\subsection{Enjeux militaires - Rétablir l'équilibre}

\section{Scénarios sur l'évolution du programme spatial chinois}

\subsection{Scénario 1: Course aux étoiles}

\subsection{Scénario 3: Statu quo}

\section{Conclusion}

\section{Bibliographie}


% TODO YANN: https://www.overleaf.com/learn/latex/Bibliography_management_in_LaTeX
\nocite{*} % displays even unused sources
\printbibliography[keyword={livre}, title={Livres}]
\printbibliography[keyword={monographie}, title={Monographies}]
\printbibliography[keyword={article}, title={Articles}]
\printbibliography[keyword={source}, title={Sources}]
\printbibliography[keyword={internet}, title={Internet}]
\printbibliography[keyword={illustration}, title={Illustrations}]
%\printbibliography[keyword={yann}, title={Yann}]
%\printbibliography[keyword={tiago}, title={Tiago}]
%\printbibliography[keyword={benno}, title={Benno}]
%\printbibliography[keyword={julien}, title={Julien}]

%\newpage
%Salut yanjboy :)

%\section*{6. Autres idées à proposer aux profs}
%\begin{enumerate}
%    \item Pourquoi tout le monde veut aller sur la Lune / Mars ?
%    \item Extraire matières premières des astéroïdes, rêve ou bientôt une réalité?
%    \item Parler des capacité hardware? Détails de programmes spatiaux, fusées, manned spaceflight, satellites ?
%\end{enumerate}

%Autres sous-sections qu'on peut ajouter:
%\begin{enumerate}
%    \item Définition d'une space race -> est-ce véritablement si négatif? (Martian -> Chine sauve astronaute américain) définir en opposant à une coopération %comme actuellement le cas (ESA, US - Russie)
%    \item Quels seraient les objets d'intérêts dans une space race (buts concrets de la missions) ? Lune, Mars, asteroids, orbit terre, soleil etc?
%    \item Aspect juridique, quels sont les lois en place en ce moment? Est-ce probable que de nouvelles traités soient ratifiés? Faut-il déjà ce poser les %questions à qui appartient la Lune par exemple?
%    \item Ou en est la Chine en ce moment ? Pleins de nouvelles impressionnantes, mais véritablement autant de capacités que les US et la Russie ? Article le %temps: les US font peur pour avoir plus de budget ! (mentionner!)
%    \item (déjà planifié) Comment vont réagir les grands aux avancés de la Chine? NASA semble être heureux, la politique pas trop. Course pour la Lune ou% coopération?
%    \item Mentionner que même au travers de crises diplomatiques, les relations dans l'espace n'ont pas vraiment souffert entre US - Russie
%    \item 
%\end{enumerate}


%\section*{Questions pour meeting}
%begin{enumerate}
 %   \item Définition d'une course à l'espace, chapitre tel quel ?
 %   \item 3 scénarios dans partie 4, comparer leur probabilité dans la conclusion ou dans une sous conclusions de chapitre ? (aka conclusions doit elle redire %des choses déjà dites ou pas?)
 %   \item Sources en notes de bas de page, ou référant à la fin ?
  %  \item Scénarios, combien? Trois scénarios assez précis, comparer leur probabilité dans la conclusion?
%\end{enumerate}
%coopération chine (uniquement si profite à la chine)
%2049 objectifs centenaire
%opacité
%enjeux réels de la chine (affirmés publiquement ou supposés ?)
%critique de nos sources (US vs Chine)%

% 
% TODO add oa picture of a chinese rocket.

% notes yann: domaine privée, a toujours été important pour usa (rockwell -> boing), ils ont pas de retard?, ASAT metnionné 3 fois, course à l'espace mentionné trop de fois, inconsistance dates, course espace fin/départ, début station spatiale ?
% - Partie historique dans Economie -> est-ce que on répète trop la partie historique?


\end{document}
